\thispagestyle{empty}

\begin{center}
{\large\bfseries \titulo \\ \subtitulo }\\
\end{center}
\begin{center}
\minombre\\
\end{center}

%\vspace{0.7cm}

\vspace{0.5cm}
\noindent\textbf{Palabras clave}: \textit{galería de arte, aplicación web, React, NodeJS, MySQL, Docker}
\vspace{0.7cm}

\noindent\textbf{Resumen}

\bigskip
El objetivo de este trabajo es el desarrollo de una aplicación web
que sirva como una galería de arte online en la que los artistas podrán publicar
sus obras para darlas a conocer. Los usuarios podrán ver y valorar las obras para poder
dar feedback a los artistas. El propósito de esta aplicación es que los artistas puedan
dar a conocer sus obras y que galerías de arte convencionales puedan descubrir nuevos
artistas para exponer sus obras.

\cleardoublepage

\begin{center}
	{\large\bfseries \titulo \\ \subtituloingles }\\
\end{center}
\begin{center}
	\minombre\\
\end{center}
\vspace{0.5cm}
\noindent\textbf{Keywords}: \textit{art gallery, web app, React, NodeJS, MySQL, Docker}
\vspace{0.7cm}

\noindent\textbf{Abstract}

\bigskip
The objective of this work is the development of a web application
that will serve as an online art gallery that serves as an online art
gallery where artists can publish their works to make them known.
Users will be able to view and rate the artworks in order to give
feedback to the artists. The purpose of this application is for artists
to be able to to make their works known and for conventional art
galleries to discover new artists to exhibit their works.


\cleardoublepage

\thispagestyle{empty}

\noindent\rule[-1ex]{\textwidth}{2pt}\\[4.5ex]

D. \textbf{\tutor}, catedrático del departamento de Ciencias de la
Computación e Inteligencia Artificial de la Universidad de Granada.

\vspace{0.5cm}

\textbf{Informo:}

\vspace{0.5cm}

Que el presente trabajo, titulado \textit{\textbf{\titulo}},
ha sido realizado bajo mi supervisión por \textbf{\minombre}, y autorizo la defensa de dicho trabajo ante el tribunal
que corresponda.

\vspace{0.5cm}

Y para que conste, expiden y firman el presente informe en Granada a 30 de Junio de 2023.

\vspace{1cm}

\textbf{El director: }

\vspace{5cm}

\noindent \textbf{\tutor}

\chapter*{Agradecimientos}




