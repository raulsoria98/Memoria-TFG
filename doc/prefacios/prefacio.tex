\thispagestyle{empty}

\begin{center}
{\large\bfseries \titulo \\ \subtitulo }\\
\end{center}
\begin{center}
\minombre\\
\end{center}

%\vspace{0.7cm}

\vspace{0.5cm}
\noindent\textbf{Palabras clave}: \textit{galería de arte, aplicación web, React,
NodeJS, MySQL, Docker, Modelo-Vista-Controlador, MVC, full-stack, ingeniería del software,
ingeniería}
\vspace{0.7cm}

\noindent\textbf{Resumen}

\bigskip
Hoy en día existe una \textbf{gran barrera} para los artistas independientes que quieren
\textbf{darse a conocer}, encontrando limitaciones tanto en las galerías de arte
convencionales como en el mundo digital. Las primeras no suelen fijarse en ellos y no les
dan la oportunidad de exponer sus obras. Además, en \textbf{el mundo digital} no hay un
gran catálogo de aplicaciones web que permitan a los artistas independientes publicar sus
obras para darse a conocer. La mayoría de aplicaciones web que existen están
\textbf{enfocadas a la venta} de obras de arte y \textbf{no tanto a la exposición} de
obras y promoción de los artistas.

El \textbf{objetivo} de este trabajo es \textbf{romper esa barrera} desarrollando
aplicación web que sirva como una \textbf{galería de arte online} en la que los artistas
puedan publicar sus obras para darlas a conocer. Los usuarios podrán ver y valorar las
obras para que después, las galerías de arte o gente interesada pueda \textbf{descubrir a
nuevos artistas} y sus obras, pudiendo ver también lo que opina la gente sobre ellas.

Para ello se ha desarrollado una aplicación web utilizando el patrón de arquitectura
\textbf{Modelo-Vista-Controlador} (MVC) y tecnologías como \textbf{React},
\textbf{NodeJS}, \textbf{MySQL} y \textbf{Docker}. Se ha realizado todo el proceso de
ingeniería del software, desde la planificación hasta la implementación de la aplicación.

\cleardoublepage

\begin{center}
	{\large\bfseries \titulo \\ \subtituloingles }\\
\end{center}
\begin{center}
	\minombre\\
\end{center}
\vspace{0.5cm}
\noindent\textbf{Keywords}: \textit{art gallery, web app, React, NodeJS, MySQL, Docker,
Model-View-Controller, MVC, full-stack, software engineering, engineering}
\vspace{0.7cm}

\noindent\textbf{Abstract}

\bigskip
Nowadays there is a \textbf{big barrier} for independent artists who want to
\textbf{make themselves known}, finding limitations both in conventional art galleries and
in the digital world. The former do not usually pay attention to them and do not give them
the opportunity to exhibit their works. In addition, in \textbf{the digital world} there
is not a large catalog of web applications that allow independent artists to publish their
works to make themselves known. Most of the existing web applications are
\textbf{focused on the sale} of artworks and \textbf{not so much on the exhibition}
of works and promotion of artists.

The \textbf{objective} of this work is to \textbf{break that barrier} by developing
a web application that serves as an \textbf{online art gallery} in which artists can
publish their works to make them known. Users will be able to see and rate the works so
that later, art galleries or interested people can \textbf{discover new artists} and their
works, being able to see also what people think about them.

For this purpose, a web application has been developed using the \textbf{Model-View-Controller}
(MVC) architecture pattern and technologies such as \textbf{React}, \textbf{NodeJS},
\textbf{MySQL} and \textbf{Docker}. The entire software engineering process has been
carried out, from planning to implementation of the application.


\cleardoublepage

\thispagestyle{empty}

\noindent\rule[-1ex]{\textwidth}{2pt}\\[4.5ex]

D. \textbf{\tutor}, catedrático del departamento de Ciencias de la
Computación e Inteligencia Artificial de la Universidad de Granada.

\vspace{0.5cm}

\textbf{Informo:}

\vspace{0.5cm}

Que el presente trabajo, titulado \textit{\textbf{\titulo}},
ha sido realizado bajo mi supervisión por \textbf{\minombre}, y autorizo la defensa de dicho trabajo ante el tribunal
que corresponda.

\vspace{0.5cm}

Y para que conste, expiden y firman el presente informe en Granada a 10 de julio de 2023.

\vspace{1cm}

\textbf{El director: }

\vspace{5cm}

\noindent \textbf{\tutor}

\chapter*{Agradecimientos}

{ \setlength{\parskip}{7mm} % Customize the space between paragraphs
En primer lugar, agradecer a mi tutor \tutor{} por su ayuda y paciencia durante la realización
de este proyecto.

También quiero agradecer a mis amigos por su apoyo y ayuda, y por haber soportado mis
idas y venidas durante la realización de este proyecto.

Gracias a Jose Manuel, mi psicólogo, por haberme ayudado en el proceso y por haberme
enseñado a gestionar mis emociones para poder continuar con el proyecto.

Gracias a mis padres y mi hermana por su apoyo y por haberme ayudado a llegar hasta aquí.

Gracias a mi abuela Pepi por ser como es y por seguir aguantando.

También, quiero agradecerme a mí mismo por no haber tirado la toalla y haberlo logrado a
pesar de los momentos difíciles.

Por útlimo, darte las gracias a ti, Celia. Me has dado gran parte de las ideas para este
proyecto, pero, sobre todo, te estoy agradecido por todo ese apoyo incondicional que me has
brindado, por esas fuerzas que me has transmitido en todo momento para seguir adelante.
Gracias por impedir que me rindiera y por estar siempre ahí para animarme y guiarme, siendo
ese faro que sólo alumbra hacia el frente. Sin ti no habría sido posible. Te quiero, Celia.
}