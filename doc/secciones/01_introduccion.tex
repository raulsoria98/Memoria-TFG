\chapter{Introducción}

La tecnología está presente en casi todos los aspectos de nuestra vida a día de hoy. De hecho,
es un recurso que permite conectarnos con realidades que, de otra manera, quedarían
invisibilizadas. Sin embargo, en el mundo del arte nos encontramos con una circunstancia algo
diferente, ya que no se suele promocionar en los medios de comunicación a los que accede la
gran mayoría de la población. En este sentido, aunque entre sus amantes sí que existe un
mayor interés por seguir conociendo nuevos artistas, quizá para el resto de la población
general queda al margen. De esta manera, si una persona no tiene la iniciativa suficiente
como para conectarse día a día a las novedades, puede resultarle más complejo mantenerse
actualizado. 
 
\section{Motivación}  
 
En este contexto, surge la idea de desarrollar una aplicación web que permita a los artistas
publicar sus obras y promocionar su trabajo. Así, cualquier persona puede tener un fácil
acceso a las mismas, aumentando así la probabilidad de que más población se acerque a este
mundo. Además, para todos aquellos amantes del arte también es una forma de, en definitiva,
mejorar y reforzar la cercanía con los autores. Por ende, la principal motivación que inicia
este proyecto es crear una plataforma que permita promocionar las obras de pequeños artistas,
que sea intuitiva y fácil de usar para cualquier persona (ya sea quien enseña su trabajo o
quien disfruta conociéndolo). En definitiva, lo que se desea es acercar la cultura a la
población y viceversa, a través del uso de la tecnología.  
 
\section{Descripción del problema} 
 
Los artistas independientes suelen tener dificultades para darse a conocer y las herramientas
que tienen a su disposición para ello, no son específicas para este fin. La mayoría de
artistas utilizan redes sociales como \textit{Instagram} \cite{instagram}, \textit{Facebook}
\cite{facebook} o \textit{Twitter} \cite{twitter} para publicar sus obras. Sin embargo, estas
aplicaciones no están diseñadas con este propósito; la mayoría de usuarios no las utilizan
para descubrir nuevos artistas sino para otros fines. Así, es difícil encontrar obras entre
la gran cantidad de contenido que se publica en estas redes.  
 
Por otro lado, las galerías de arte convencionales o estudios fotográficos no cuentan con
herramientas para poder descubrir nuevos artistas en sus exposiciones. En la mayoría de casos,
los artistas que muestran su trabajo en estos lugares ya tienen cierto reconocimiento y se
han dado a conocer anteriormente en otras galerías. Sin embargo, aquellos que se están
iniciando en este mundo y que quizá no han tenido las mismas oportunidades y/o facilidades
que sus colegas para acceder a estos recursos, a veces simplemente por mera cuestión de
suerte, tienen más dificultades para empezar a trabajar de lo que tanto les gusta.  

Además, los grandes centros urbanos suelen ser los lugares que alojan eventos culturales para
promocionar el arte independiente. Sin embargo, a día de hoy es más difícil encontrar esta
misma situación en ciudades y pueblos más pequeños. En ellos, el único medio que tienen las
personas para conocer nuevos artistas es a través de internet, pero, como se ha descrito
anteriormente, no existe ninguna plataforma específica para este fin.

\section{Objetivos}

El propósito de este proyecto es el desarrollo de una aplicación web que sirva como una
galería de arte online en la que los artistas puedan publicar sus obras para darlas a
conocer. Los usuarios podrán ver y valorar las obras para poder dar feedback a los
artistas sobre sus trabajos. El objetivo de esta aplicación es que los artistas puedan
dar a conocer sus obras y que las galerías de arte convencionales puedan descubrir nuevos
artistas para exponer sus obras.

\section{Revisión del estado del Arte}

Existen varias aplicaciones web que sirven como galerías de arte online, pero la mayoría
de ellas están enfocadas a la venta de obras de arte y no tanto a la exposición de obras
y promoción de las mismas.

Un ejemplo de esto es la aplicación web

\section{Estructura de la memoria}
