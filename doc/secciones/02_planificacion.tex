\chapter{Planificación}
En este capítulo se va a realizar una planificación del TFG, en la que se
va a explicar la metodología que será utilizada para el desarrollo del proyecto,
se va a estimar el tiempo que se va a tardar en realizar cada tarea del proyecto
y el coste económico que supondría realizar el mismo.

\section{Metodología utilizada}
Para el desarrollo de la aplicación voy a enfocarlo con una \textbf{metodología
tradicional}, ya que es la que más cómoda me resulta y la que más se adapta
a este tipo de proyectos. Además, al ser un proyecto de una sola persona, no es
necesario utilizar una metodología ágil y desde mi punto de vista entorpecería
más el flujo de trabajo.

No es necesario utilizar una metodología ágil porque no hay un equipo de
desarrollo que coordinar en daily meetings, las tareas se realizarán de forma secuencial.
No hay un cliente o stakeholders que puedan cambiar los requisitos de forma
continua, por lo que no es necesario realizar entregas periódicas en sprints como
sí se hace en metodologías ágiles como Scrum. En nuestro caso no habrá feedback del
cliente ni reuniones de retroalimentación, sino que será una única persona la que
se encargará de realizar el control de calidad y verificar que se cumplan los
requisitos de la aplicación.

Para el control de calidad, pruebas y verificación de requisitos, se irá haciendo
de manera continua a medida que se vayan desarrollando las funcionalidades de la
aplicación.

\subsection{Seguimiento del proyecto}
Para el seguimiento del desarrollo de la aplicación se utilizará la herramienta
\textit{Trello} \cite{trello}, perteneciente a la empresa \textit{Atlassian} \cite{atlassian}.
Esta herramienta permite crear tableros de kanban para oraganizar
las tareas pendientes de realizar, las que están en proceso y las que ya se han
realizado. Se puede ver el tablero kanban en la figura \ref{fig:trello}.

En la imágen se puede ver cómo están dividadas las tareas en las tres columnas y
etiquetadas según la etapa a la que pertenecen. Algunas de las tarjetas de tarea
tienen checklist para poder ir marcando las subtareas que se van realizando. Esto
permite tener un seguimiento de las tareas que se van realizando y las que aún
faltan por hacer.

\begin{figure}[H]
  \centering
  \includegraphics[width=1\textwidth]{trello}
  \caption{Tablero kanban de Trello}
  \label{fig:trello}
\end{figure}

\section{Control de versiones}
Para realizar un seguimiento del desarrollo de la aplicación se utilizará la
herramienta de control de versiones \textit{Git} \cite{git}, que permite llevar un
control de los cambios realizados en el código fuente de la aplicación. Además, estos
cambios se irán subiendo a la plataforma \textit{GitHub} \cite{github}.

Para seguir una estructura de los commits que se van realizando, se utilizará
el mensaje de commit con el formato "\texttt{<stack>} - \texttt{<mensaje>}", donde \texttt{<stack>}
es \textit{SERVER} o \textit{CLIENT} dependiendo de si el commit afecta al backend o
al frontend de la aplicación, y \texttt{<mensaje>} es la descripción de los cambios que
se han realizado.

En la figura \ref{fig:commits} se puede ver un ejemplo de una captura sacada
del listado de commits realizados en el repositorio de \textit{GitHub} \cite{github}.

\begin{figure}[H]
  \centering
  \includegraphics[width=1\textwidth]{commits}
  \caption{Ejemplo de mensajes de commit}
  \label{fig:commits}
\end{figure}

\section{Temporización}
En esta sección se mostrará a través de un diagrama de Gantt la planificación
temporal del proyecto desglosado en tareas. Para ello se ha utilizado la
herramienta \textit{Lucidchart} \cite{lucidchart}.

En la figura \ref{fig:gantt} se muestra el diagrama de Gantt con el desglose de las
tareas que se van a realizar, así como la duración de cada una de ellas en semanas.
Los colores diferencian las etapas del desarrollo del proyecto.
\begin{figure}[H]
  \centering
  \includegraphics[width=1\textwidth]{gantt}
  \caption{Diagrama de Gantt}
  \label{fig:gantt}
\end{figure}

\section{Presupuesto}
Para elaborar el presupuesto se ha tenido en cuenta el coste de la hora de trabajo
de un ingeniero informático en España, que es de unos 14€/hora \cite{coste-hora}.
Teniendo en cuenta que la duración del proyecto se estima en 16 semanas, y que
se van a dedicar unas 20 horas semanales, el coste del personal sería de \textbf{4480€}.

\vspace{1cm}

En cuanto al coste de los recursos materiales, se ha calculado el coste del portátil
prorrateado en el tiempo que se va a utilizar para el desarrollo del proyecto. El
coste del portátil es de 1799€ \cite{portatil} y teniendo en cuenta que el ciclo de vida de un
portátil es de unos 5 años, el coste prorrateado sería de 29,98€/mes. Por lo tanto,
el coste del portátil durante las 16 semanas de duración del proyecto sería de
\textbf{119,92€}.

\vspace{1cm}

Por último, el coste de los recursos de software utilizados para el desarrollo
del proyecto es de \textbf{0€}, ya que se han utilizado herramientas de software libre.

El editor de texto utilizado es \textit{Visual Studio Code} \cite{vscode} que tiene
licencia \textit{MIT} \cite{mit-license}.

El sistema de control de versiones \textit{Git} \cite{git} tiene licencia
\textit{GPLv2} \cite{gplv2}. Y la plataforma \textit{GitHub} \cite{github} tiene
licencia \textit{MIT} \cite{mit-license}.

Para el desarrollo del backend se va a utilizar \textit{Node.js} \cite{nodejs}
y para el frontend \textit{React} \cite{react}, ambos con licencia
\textit{MIT} \cite{mit-license} también.

\vspace{1cm}

Teniendo en cuenta todo lo anterior, el coste total del proyecto sería de
\textbf{4599,92€}.
