\chapter{Análisis del problema}
En este capítulo vamos a realizar un análisis de las funcionalidades que debe cumplir
nuestra aplicación. Para ello, vamos a definir los requisitos funcionales,
no funcionales y de información que debe cumplir. Además, vamos a definir los
casos de uso que se van a dar en nuestra aplicación con sus respectivos diagramas.

\section{Requisitos}
Los actores que van a interactuar con nuestra aplicación son los siguientes:

\begin{itemize}
    \item \textbf{Usuario:} Usuario no registrado en la aplicación. Tendrá funcionalidades
    básicas.
    \item \textbf{Usuario registrado:} Usuario registrado en la aplicación. Tendrá
    acceso a más funcionalidades que el usuario no registrado.
    \item \textbf{Artista:} Usuario registrado con rol de artista. Podrá subir sus
    obras a la aplicación.
\end{itemize}

\subsection{Requisitos funcionales}
A continuación se van a definir los requisitos funcionales que debe cumplir nuestra
aplicación.

\begin{itemize}
    \item \textbf{RF-1:} Registrarse en la aplicación
    \begin{itemize}
        \item \textbf{RF-1.1:} Registrarse como usuario
        \item \textbf{RF-1.2:} Registrarse como artista
    \end{itemize}
    \item \textbf{RF-2:} Iniciar sesión
    \item \textbf{RF-3:} Eliminar usuario
    \begin{itemize}
        \item \textbf{RF-3.1:} Borrar cuenta como usuario registrado
        \item \textbf{RF-3.2:} Borrar cuenta como administrador
    \end{itemize}
    \item \textbf{RF-4:} Mostrar galería
    \item \textbf{RF-5:} Subir obra en galería
    \item \textbf{RF-6:} Mostrar detalles de obra
    \item \textbf{RF-7:} Valorar obra
    \item \textbf{RF-8:} Mostrar valoraciones de obra
    \item \textbf{RF-9:} Ordenar obras por fecha
    \begin{itemize}
        \item \textbf{RF-9.1:} Ordenar obras fecha descendente
        \item \textbf{RF-9.2:} Ordenar obras fecha ascendente
    \end{itemize}
    \item \textbf{RF-10:} Ordenar obras por valoración
    \begin{itemize}
        \item \textbf{RF-10.1:} Ordenar obras valoración descendente
        \item \textbf{RF-10.2:} Ordenar obras valoración ascendente
    \end{itemize}
    \item \textbf{RF-11:} Filtrar obras por autor
    \item \textbf{RF-12:} Eliminar obra
    \begin{itemize}
        \item \textbf{RF-12.1:} Eliminar obra como artista
        \item \textbf{RF-12.2:} Eliminar obra como administrador
    \end{itemize}
    \item \textbf{RF-13:} Eliminar valoración
    \begin{itemize}
        \item \textbf{RF-13.1:} Eliminar valoración como usuario registrado
        \item \textbf{RF-13.2:} Eliminar valoración como administrador
    \end{itemize}
    \item \textbf{RF-14:} Cerrar sesión
\end{itemize}

\subsection{Requisitos no funcionales}
En cuanto a los requisitos no funcionales, se van a definir los siguientes:

\begin{itemize}
    \item \textbf{RNF-1:} Eficiencia
    \begin{itemize}
        \item \textbf{RNF-1.1:} Toda funcionalidad ha de responder al usuario en menos
        de 5 segundos.
        \item \textbf{RNF-1.2:} La aplicación ha de operar correctamente con hasta 100
        sesiones concurrentes.
        \item \textbf{RNF-1.3:} Los datos modificados en la base de datos tienen que
        estar actualizados en menos de 5 segundos.
    \end{itemize}
    \item \textbf{RNF-2:} Seguridad lógica y de datos
    \begin{itemize}
        \item \textbf{RNF-2.1:} Los permisos de usuario administarador solo pueden ser
        modificados por otro usuario administrador.
        \item \textbf{RNF-2.2:} La aplicación se desarrollará aplicando patrones de
        programación que aumenten la seguirdad de los datos.
        \item \textbf{RNF-2.3:} Los datos sensibles de los usuarios serán encriptados
        antes de almacenarse en la base de datos.
    \end{itemize}
    \item \textbf{RNF-3:} Usabilidad
    \begin{itemize}
        \item \textbf{RNF-3.1:} La aplicación ha de ser intuitiva y fácil de usar por
        personas que no posean grandes conocimientos informáticos.
        \item \textbf{RNF-3.2:} La aplicación cumplirá con los estándares mínimos de
        accesibilidad.
        \item \textbf{RNF-3.3:} Los mensajes de error que proporcione la aplicación
        han de ser informativos y orientados al usuario final.
        \item \textbf{RNF-3.4:} La aplicación ha de ser compatible con los navegadores
        más utilizados.
        \item \textbf{RNF-3.5:} La aplicación ha de tener un diseño responsive, adaptándose
        a diversos tamaños de pantalla y dispositivos.
        \item \textbf{RNF-3.6:} Las imágenes se almacenarán en formato \textit{JPEG} o
        \textit{PNG}.
    \end{itemize}
\end{itemize}

\subsection{Requisitos de información}
Ahora se van a definir los requisitos de información que debe cumplir nuestra
aplicación.

\begin{itemize}
    \item \textbf{RI-1:} Usuario
    \begin{itemize}
        \item \textbf{RI-1.1:} Nombre
        \item \textbf{RI-1.2:} Apellidos
        \item \textbf{RI-1.3:} Correo electrónico
        \item \textbf{RI-1.4:} Contraseña
        \item \textbf{RI-1.5:} Es o no artista
        \item \textbf{RI-1.6:} Es o no administrador
    \end{itemize}
    \item \textbf{RI-2:} Obra
    \begin{itemize}
        \item \textbf{RI-2.1:} Título
        \item \textbf{RI-2.2:} Descripción
        \item \textbf{RI-2.3:} Fecha de creación
        \item \textbf{RI-2.4:} Valoración media
        \item \textbf{RI-2.5:} Autor
        \item \textbf{RI-2.6:} Archivo
    \end{itemize}
    \item \textbf{RI-3:} Valoración
    \begin{itemize}
        \item \textbf{RI-3.1:} Valoración
        \item \textbf{RI-3.2:} Comentario
    \end{itemize}
\end{itemize}

\subsection{Descripción de los requisitos}
A continuación se van a realizar las tablas de descripción de los requisitos funcionales.

\begin{table}[H]
    \centering
    \begin{tabular}{|p{4cm}|p{7cm}|}
        \hline
        \rowcolor{lightgray}
        \textbf{Detalle} & \textbf{Descripción} \\
        \hline
        \textbf{RF\#} & 1.1 \\
        \hline
        \textbf{Nombre} & Registrarse como usuario registrado \\
        \hline
        \textbf{Descripción} & Un usuario podrá registrarse en la aplicación \\
        \hline
        \textbf{Entrada} &
        Agente externo: Usuario
        
        Requisito de datos de entrada: RDE-1.1 \\
        \hline
        \textbf{BD} &
        Requisito de datos de lectura: ninguno
        
        Requisito de datos de escritura: RDW-1.1 \\
        \hline
        \textbf{Salida} & Requisito de datos de salida: ninguno \\
        \hline
        \textbf{RDE-1.1} & Datos de entrada para el registro de un usuario:
            \begin{itemize}
                \item Datos de usuario RI-1
            \end{itemize} \\
        \hline
        \textbf{RDW-1.1} & Datos almacenados de un usuario:
            \begin{itemize}
                \item Datos de usuario RI-1
            \end{itemize} \\
        \hline
    \end{tabular}
    \caption{Descripción del requisito funcional RF-1.1}
    \label{tab:rf-1-1}
\end{table}

\begin{table}[H]
    \centering
    \begin{tabular}{|p{4cm}|p{7cm}|}
        \hline
        \rowcolor{lightgray}
        \textbf{Detalle} & \textbf{Descripción} \\
        \hline
        \textbf{RF\#} & 1.2 \\
        \hline
        \textbf{Nombre} & Registrarse como artista \\
        \hline
        \textbf{Descripción} & Un usuario podrá registrarse en la aplicación como artista \\
        \hline
        \textbf{Entrada} &
        Agente externo: Usuario
        
        Requisito de datos de entrada: RDE-1.2 \\
        \hline
        \textbf{BD} &
        Requisito de datos de lectura: ninguno
        
        Requisito de datos de escritura: RDW-1.2 \\
        \hline
        \textbf{Salida} & Requisito de datos de salida: ninguno \\
        \hline
        \textbf{RDE-1.2} & Datos de entrada para el registro de un usuario como artista:
            \begin{itemize}
                \item Usuario RI-1
            \end{itemize} \\
        \hline
        \textbf{RDW-1.2} & Datos almacenados de un usuario:
            \begin{itemize}
                \item Usuario RI-1
            \end{itemize} \\
        \hline
    \end{tabular}
    \caption{Descripción del requisito funcional RF-1.2}
    \label{tab:rf-1-2}
\end{table}

\begin{table}[H]
    \centering
    \begin{tabular}{|p{4cm}|p{7cm}|}
        \hline
        \rowcolor{lightgray}
        \textbf{Detalle} & \textbf{Descripción} \\
        \hline
        \textbf{RF\#} & 2 \\
        \hline
        \textbf{Nombre} & Iniciar sesión \\
        \hline
        \textbf{Descripción} & Un usuario podrá iniciar sesión en la aplicación \\
        \hline
        \textbf{Entrada} &
        Agente externo: Usuario
        
        Requisito de datos de entrada: RDE-2 \\
        \hline
        \textbf{BD} &
        Requisito de datos de lectura: RDR-2
        
        Requisito de datos de escritura: ninguno \\
        \hline
        \textbf{Salida} & Requisito de datos de salida: ninguno \\
        \hline
        \textbf{RDE-2} & Datos de entrada para iniciar sesión:
            \begin{itemize}
                \item Correo electrónico RI-1.3
                \item Contraseña RI-1.4
            \end{itemize} \\
        \hline
        \textbf{RDR-2} & Datos de lectura de un usuario:
            \begin{itemize}
                \item Usuario RI-1
            \end{itemize} \\
        \hline
    \end{tabular}
    \caption{Descripción del requisito funcional RF-2}
    \label{tab:rf-2}
\end{table}

\begin{table}[H]
    \centering
    \begin{tabular}{|p{4cm}|p{7cm}|}
        \hline
        \rowcolor{lightgray}
        \textbf{Detalle} & \textbf{Descripción} \\
        \hline
        \textbf{RF\#} & 3.1 \\
        \hline
        \textbf{Nombre} & Borrar cuenta como usuario registrado \\
        \hline
        \textbf{Descripción} & Un usuario registrado podrá borrar su cuenta \\
        \hline
        \textbf{Entrada} &
        Agente externo: Usuario registrado
        
        Requisito de datos de entrada: ninguno \\
        \hline
        \textbf{BD} &
        Requisito de datos de lectura: RDR-3.1
        
        Requisito de datos de escritura: RDW-3.1 \\
        \hline
        \textbf{Salida} & Requisito de datos de salida: ninguno \\
        \hline
        \textbf{RDR-3.1} & Datos de lectura para borrar un usuario:
            \begin{itemize}
                \item Usuario RI-1
                \item Obras del usuario RI-2
            \end{itemize} \\
        \hline
        \textbf{RDW-3.1} & Datos de escritura de un usuario:
            \begin{itemize}
                \item Eliminar usuario RI-1
                \item Eliminar obras del usuario RI-2
            \end{itemize} \\
        \hline
    \end{tabular}
    \caption{Descripción del requisito funcional RF-3.1}
    \label{tab:rf-3-1}
\end{table}

\section{Casos de uso}

\section{Diagramas de secuencia}
