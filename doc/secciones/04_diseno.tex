\chapter{Diseño}
En este capítulo se abordará el diseño del sistema. Para ello, en primer
lugar, se va a realizar el diseño de la base de datos con el modelo
entidad-relación, el paso a tablas y la normalización. En segundo lugar,
se realizará el diseño de las interfaces de usuario. Por se hablará del
diseño de la arquitectura del sistema.

\section{Diseño de base de datos}
En esta sección vamos a abordar el diseño de la base de datos.
Comenzaremos con el modelo entidad-relación y terminaremos con el paso a
tablas y la normalización.

\subsection{Modelo Entidad-Relación}
En nuestro caso, la base de datos va a estar formada por las siguientes
entidades:

\begin{itemize}
    \item \textbf{Usuario}: Esta entidad representa a los usuarios de la aplicación.
    \item \textbf{Artista}: Esta entidad representa a los artistas de la aplicación.
    \item \textbf{ObraDeArte}: Esta entidad representa a las obras de arte de la aplicación.
    \item \textbf{Valoración}: Esta entidad representa a las valoraciones de las obras de
    arte que realizan los usuarios.
\end{itemize}

En la figura \ref{fig:e-r} se muestra el modelo entidad-relación de la base de
datos. En dicho modelo se encuentran las entidades con sus atributos y las
relaciones entre ellas.

Vemos que la entidad \texttt{Artista} hereda de la entidad \texttt{Usuario}.
Esto se hace para poder relacionar la entidad \texttt{Artista} con la entidad
\texttt{ObraDeArte} a través de la relación \texttt{Tiene} y que sea independiente
de la entidad \texttt{Usuario} ya que solamente los usuarios que sean artistas
podrán tener obras de arte.

La entidad \texttt{Usuario} tiene una relación \texttt{Realiza} con la entidad
\texttt{Valoración} indicando que un usuario puede realizar varias valoraciones.
A su vez la entidad \texttt{Valoración} tiene una relación \texttt{Pertenece} con
la entidad \texttt{ObraDeArte} indicando que una valoración pertenece a una obra
de arte.

\begin{figure}[H]
  \centering
  \includegraphics[width=\textwidth]{diagramas/e-r}
  \caption{Modelo entidad-relación}
  \label{fig:e-r}
\end{figure}

\subsection{Paso a tablas}
A continuación vamos a realizar el paso a tablas del modelo entidad-relación. Para
ello haremos las tablas primero de las entidades fuertes, luego entidades débiles
y por último las relaciones.

\begin{enumerate}
    \item \[ \texttt{Usuario}(\underline{ID-Usuario}_{CP}, \underline{Email}_{CC}, Nombre, Contraseña) \]
    \item \[ \texttt{Artista}(\underline{ID-Usuario}_{CP}^{CE1}) \]
    \item \[ \texttt{ObraDeArte}(\underline{ID-Obra}_{CP}, Titulo, Descripción, Tipo, Archivo) \]
    \item \[ \texttt{Valoración}(\underline{ID-Valoración}_{CP}, Puntuación, Comentario) \]
    \item \[ \texttt{Tiene}(\underline{ID-Obra}_{CP}^{CE3}, ID-Usuario^{CE2}) \]
    \item \[ \texttt{Pertenece}(\underline{ID-Valoración}_{CP}^{CE4}, ID-Obra^{CE3}) \]
    \item \[ \texttt{Realiza}(\underline{ID-Valoración}_{CP}^{CE4}, ID-Usuario^{CE1}) \]
\end{enumerate}

\vspace{1cm}
Ahora se va a proceder a fusionar las tablas que sean posibles. Además,
dado que la tabla \texttt{Artista} no tiene atributos propios y solamente tiene
una clave externa que es clave primaria en la tabla \texttt{Usuario} (que es la
tabla padre), se puede fusionar con dicha tabla, añadiendo a esta una columna
que indique si el usuario es artista o no, será el atributo \texttt{Rol}.

\begin{enumerate}
    \item \[ \texttt{Usuario}(\underline{ID-Usuario}_{CP}, \underline{Email}_{CC}, Nombre, Contraseña, Rol) \]
    \item \[ \texttt{ObraDeArte}(\underline{ID-Obra}_{CP}, Titulo, Descripción, Tipo, Archivo, ID-Usuario^{CE1}) \]
    \item \[ \texttt{Valoración}(\underline{ID-Valoración}_{CP}, Puntuación, Comentario, ID-Obra^{CE2}, ID-Usuario^{CE1}) \]
\end{enumerate}

\subsection{Normalización}
Dado que todos los atributos de las tablas están en 1FN, vamos a proceder a
normalizar las tablas a 2FN y 3FN.

\subsubsection{2FN}
Para que una tabla esté en 2FN debe estar en 1FN y además todos los atributos
que no pertenezcan a la clave primaria deben depender de esta.

En nuestro caso, todas las tablas están en 1FN y además todos los atributos
dependen de la clave primaria, por lo que todas las tablas están en 2FN.

\subsubsection{3FN}
Para que una tabla esté en 3FN debe estar en 2FN y además todos los atributos
que no pertenezcan a la clave primaria deben depender únicamente de dicha
clave primaria y no de otros atributos.

En nuestro caso, todas las tablas están en 2FN y además todos los atributos
dependen de la clave primaria y no de otros atributos no-clave, por lo que
todas las tablas están en 3FN.

\newpage

\section{Diseño de interfaces de usuario}
En esta sección se va a realizar el diseño de las interfaces de usuario, para
ello se van a realizar bocetos de las mismas. Estos bocetos se han realizado
utilizando la herramienta \textit{Sketchbook} \cite{sketchbook}.

\subsection{Bocetos}
El primer boceto que se va a realizar es el de la interfaz de inicio de sesión
\ref{fig:boceto-iniciar-sesion}. Cumpliendo así con el requisito funcional
\hyperref[tab:rf-2]{\textbf{RF-2}}.

\begin{figure}[H]
  \centering
  \includegraphics[width=\textwidth]{img/iniciar-sesion}
  \caption{Boceto de la interfaz de inicio de sesión}
  \label{fig:boceto-iniciar-sesion}
\end{figure}

El siguiente boceto que se va a realizar es el de la interfaz de registro
\ref{fig:boceto-registro}. Cumpliendo así con los requisitos funcionales
\hyperref[tab:rf-1-1]{\textbf{RF-1.1}} y \hyperref[tab:rf-1-2]{\textbf{RF-1.2}}.

\begin{figure}[H]
  \centering
  \includegraphics[width=\textwidth]{img/registrarse}
  \caption{Boceto de la interfaz de registro}
  \label{fig:boceto-registro}
\end{figure}

A continuación, se va a realizar el boceto de la interfaz de una galería de arte
\ref{fig:boceto-galeria}. Cumpliendo así con el requisito funcional
\hyperref[tab:rf-4]{\textbf{RF-4}}.

\begin{figure}[H]
  \centering
  \includegraphics[width=\textwidth]{img/galeria}
  \caption{Boceto de la interfaz de una galería de arte}
  \label{fig:boceto-galeria}
\end{figure}

El siguiente boceto que se va a realizar es el de el modal con los detalles de una obra de
arte \ref{fig:boceto-obra}, donde también se puede valorar la misma. Cumpliendo así con el
requisito funcional \hyperref[tab:rf-6]{\textbf{RF-6}} referido a mostrar los detalles de una
obra de arte y con el requisito funcional \hyperref[tab:rf-7]{\textbf{RF-7}} referido a
valorar una obra.

\begin{figure}[H]
  \centering
  \includegraphics[width=\textwidth]{img/modal}
  \caption{Boceto del modal con los detalles de una obra de arte}
  \label{fig:boceto-obra}
\end{figure}

Proseguimos con el boceto del modal de valoraciones de una obra de arte
\ref{fig:boceto-valoraciones}. Cumpliendo así con el requisito funcional
\hyperref[tab:rf-8]{\textbf{RF-8}}.

\begin{figure}[H]
  \centering
  \includegraphics[width=\textwidth]{img/valoraciones}
  \caption{Boceto del modal de valoraciones de una obra de arte}
  \label{fig:boceto-valoraciones}
\end{figure}

Por último, se va a realizar el boceto de la pantalla de la que disponen los artistas para
subir sus obras de arte \ref{fig:boceto-subir-obra}. Cumpliendo así con el requisito
funcional \hyperref[tab:rf-5]{\textbf{RF-5}}.

\begin{figure}[H]
  \centering
  \includegraphics[width=\textwidth]{img/subir-obra}
  \caption{Boceto de la pantalla de subir obra}
  \label{fig:boceto-subir-obra}
\end{figure}

\section{Diseño de la arquitectura del sistema}
A continuación se va a explicar la arquitecura del sistema utilizada.
Para la creación de la aplicación web se va a utilizar el patrón
Modelo-Vista-Controlador (MVC).

\subsection{Modelo-Vista-Controlador}
Este patrón de arquitectura se basa, como se explica en este artículo \cite{mvc},
en separar la lógica de la aplicación en tres partes:

\begin{itemize}
    \item \textbf{Modelo}: El modelo representa la información con la que trabaja
    la aplicación. En el caso de la aplicación que vamos a desarrollar, el modelo
    será la base de datos y las funciones del backend que se encarguen de gestionarla.
    \item \textbf{Vista}: La vista es la parte encargada de mostrar la información
    al usuario y permitir que este interactúe con la aplicación. En el caso de la
    aplicación que vamos a desarrollar, la vista será la parte del frontend.
    \item \textbf{Controlador}: El controlador es el encargado de gestionar las
    peticiones del usuario y realizar las acciones correspondientes. Es el enlace
    entre la vista y el modelo. En nuestra aplicación el controlador será la parte
    del backend que se encargue de gestionar las peticiones provenientes del frontend
    y a través de una serie de funciones \textit{getters} y \textit{setters} se
    pedirá y modificará la información del modelo.
\end{itemize}

En la figura \ref{fig:mvc} se muestra un diagrama de la arquitectura MVC de la
aplicación.

\begin{figure}[h]
  \centering
  \includegraphics[width=\textwidth]{diagramas/mvc}
  \caption{Modelo-Vista-Controlador}
  \label{fig:mvc}
\end{figure}
