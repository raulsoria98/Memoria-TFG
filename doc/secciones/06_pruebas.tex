\chapter{Pruebas}

{ \setlength{\parskip}{6mm} % Customize the space between paragraphs
Las pruebas de la aplicación se han realizado a lo largo del desarrollo de la misma. Se ha
ido probando cada funcionalidad conforme se iba implementando.

Durante el proceso de implementación había funcionalidades que se probaron y no se
encontró ningún error, pero cuando se terminó de implementar la aplicación y se volvieron
a probar todas las funcionalidades y casos de uso de la aplicación, se encontraron errores
que no se habían detectado anteriormente.

Un ejemplo de esto es el caso de uso de subir una obra de arte. Cuando se implementó
esta funcionalidad se probó y no se encontró ningún error, pero cuando se terminó de
implementar la aplicación y se volvió a probar, se encontró el error. La descripción
de la obra tenía un límite de 500 caracteres en las comprobaciones que se hacían mediante
el sistema de DTO, pero en base de datos, este límite era de 200 caracteres. Esto provocaba
que si se introducía una descripción de más de 200 caracteres, saltase un error de la base de
datos en lugar de un error personalizado de la aplicación.

Se han realizado las pruebas de todas las funcionalidades tanto conforme se iban
implementando como al finalizar la implementación de la aplicación y en la fase de
implementación sí es cierto que se encontraron bastantes errores que se fueron corrigiendo,
pero en la fase final de pruebas, estaba todo más depurado y se encontraron menos errores.

En un futuro, si la aplicación se sigue desarrollando y se lanza a producción, se deberían
realizar pruebas unitarias y de integración para asegurar que la aplicación funciona
correctamente y que no se introducen errores al realizar cambios en el código.
}