\chapter{Conclusiones y trabajos futuros}

\section{Conclusiones}

El desarrollo de este proyecto ha permitido la creación de una aplicación web que promociona
la cultura y el arte, facilitando a los artistas independientes la exposición de sus obras a
través de internet y permitiendo que los usuarios puedan descubrir nuevas obras y artistas.

\vspace{0.5cm}

En primer lugar, se han explicado los objetivos y la motivación del proyecto, así como un
estudio de las aplicaciones web existentes en el mercado que ofrecen servicios similares.
Esto ha servido para saber qué características debe tener la aplicación y si realmente
es necesaria.

\vspace{0.5cm}

En segundo lugar, se ha explicado la metodología que se seguiría en el desarrollo de la
aplicación y la planificación temporal y presupuesto del proyecto.

\vspace{0.5cm}

En tercer lugar, se ha realizado la fase de análisis, en la que se han definido los
requisitos funcionales y no funcionales de la aplicación, así como los de información.
También se ha descrito detalladamente cada uno de los requisitos funcionales explicando
los agentes que intervienen así como los requisitos de entrada, salida y almacenamiento
de cada uno de ellos.

También se han especificado los actores que intervienen en la aplicación y los casos de
uso que ha de poder realizar cada uno de ellos. Además, se han realizado los diagramas
de casos de uso y de secuencia de los casos de uso más importantes.

\vspace{0.5cm}

En cuarto lugar, se ha descrito la fase de diseño de la aplicación. En esta fase se ha
realizado el diseño de la base de datos con el modelo entidad-relación, el paso a tablas
y la normalización. También se ha realizado el diseño de las interfaces de usuario y el
diseño de la arquitectura del sistema, donde se ha explicado el patrón de diseño MVC
(Modelo-Vista-Controlador) que se ha utilizado para el desarrollo de la aplicación.

\vspace{0.5cm}

Por último, se ha realizado la fase de implementación, en la que se ha explicado qué
tecnologías se han utilizado para el desarrollo de la aplicación y por qué se han
elegido. También se ha explicadoo cómo se ha implementado la base de datos de la aplicación
mostrando algunos extractos de código. Se ha realizado también el diagrama de clases
de la aplicación explicando sus relaciones.

La implementación de la aplicación constaba de dos partes: el backend y el frontend. Se ha
explicado cómo se ha implementado cada una de ellas, mostrando algunos extractos de código
y explicando las decisiones de implementación que se han tomado.

\vspace{0.5cm}

En conclusión, se ha conseguido desarrollar una aplicación web que permite a los artistas
independientes exponer sus obras de arte y a los usuarios descubrir nuevas obras y artistas.
La aplicación cumple con todos los requisitos funcionales y no funcionales que se han
especificado en la fase de análisis.

\section{Trabajos futuros}

A continuación se muestran algunas de las posibles mejoras que se podrían realizar en
la aplicación de cara a futuras versiones:

\begin{itemize}
    \item \textbf{Añadir más tipos de obras de arte}: La aplicación contempla obras de tipo
    pintura y fotografía. En futuras versiones se podrían añadir tipos como escultura,
    narrativa, poesía, música, etc.
    \item \textbf{Confirmación de la cuenta de usuario}: Se podría implementar que al
    registrarse, el usuario tuviera que confirmar su cuenta a través de un correo electrónico.
    \item \textbf{Sistema de mensajería}: Se podría implementar un sistema de mensajería
    para que los usuarios pudieran contactar con los artistas. Facilitando así la comunicación
    entre los usuarios interesados en las obras y sus autores.
    \item \textbf{Traducciones a otros idiomas}: Traducción de la aplicación a otros idiomas
    para que los usuarios puedan elegir en qué idoma quieren ver la aplicación.
\end{itemize}